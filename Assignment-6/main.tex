\documentclass[journal,12pt,twocolumn]{IEEEtran}

\usepackage{enumitem}
\usepackage{tfrupee}
\usepackage{amsmath}
\usepackage{amssymb}
\usepackage{gensymb}
\usepackage{graphicx}
\usepackage{txfonts}

\def\inputGnumericTable{}

\usepackage[latin1]{inputenc}                                 
\usepackage{color}                                            
\usepackage{array}                                            
\usepackage{longtable}                                        
\usepackage{calc}                                             
\usepackage{multirow}                                         
\usepackage{hhline}                                           
\usepackage{ifthen}
\usepackage{caption} 
\captionsetup[table]{skip=3pt}  
\providecommand{\pr}[1]{\ensuremath{\Pr\left(#1\right)}}
\providecommand{\cbrak}[1]{\ensuremath{\left\{#1\right\}}}
\renewcommand{\thefigure}{\arabic{table}}
\renewcommand{\thetable}{\arabic{table}}                                     
                               
\title{Assignment 6 \\ \Large AI1110: Probability and Random Variables \\ \large Indian Institute of Technology Hyderabad}
\author{Ankit Saha \\ \normalsize AI21BTECH11004 \\ \vspace*{20pt} \normalsize  29 April 2022 \\ \vspace*{20pt} \Large CBSE Probability Grade 11}


\begin{document}
	% The title
	\maketitle
	
	% The question
	\textbf{Example 9} 
	Let a sample space be $S = \cbrak{\omega_1, \omega_2, \ldots, \omega_6}$. Which of the following assignments of probabilities to each outcome are valid? 
	
	% The answer
	\textbf{Solution.}
	Let a random variable $X \in \mathcal{X}$ where $\mathcal{X} = \cbrak{1,2,3,4,5,6}$ denote each of the six outcomes respectively.
	
	The necessary conditions for a given set of assignments of probabilities $\mathcal{P}$ to be valid are:
	\begin{align}
	0 \le \pr{X=i} &\le 1, ~\forall i \in \mathcal{X} \label{cond1} \\
	\sum_{i \in \mathcal{X}} \pr{X=i} &= 1 \label{cond2}
	\end{align}
	
	If either of these conditions fails, then the given assignment is invalid.
	
	\begin{enumerate}[label=(\alph*)]
	\item 
	\begin{align}
		\mathcal{P} = \cbrak{\frac16, \frac16, \frac16, \frac16, \frac16, \frac16}
	\end{align}
	Valid: Both conditions hold

	\item 
	\begin{align}
		\mathcal{P} = \cbrak{1, 0, 0, 0, 0, 0}
	\end{align}
	Valid: Both conditions hold
	
	\item 
	\begin{align}
		\mathcal{P} = \cbrak{\frac18, \frac23, \frac13, \frac13, -\frac14, -\frac13}
	\end{align}
	Invalid: Both conditions fail
	\begin{align}
		\pr{X=5} &< 0 \\
		\pr{X=6} &< 0 \\
		\sum_{i \in \mathcal{X}} \pr{X=i} &= \frac78
	\end{align}
	
	\item 
	\begin{align}
		\mathcal{P} = \cbrak{\frac{1}{12}, \frac{1}{12}, \frac16, \frac16, \frac16, \frac32}
	\end{align}
	Invalid: Both conditions fail	
	\begin{align}
		\pr{X=6} &> 1 \\
		\sum_{i \in \mathcal{X}} \pr{X=i} &= \frac{13}{6}
	\end{align}
	
	\item 
	\begin{align}
		\mathcal{P} = \cbrak{0.1, 0.2, 0.3, 0.4, 0.5, 0.6}
	\end{align}
	Invalid: Condition \eqref{cond2} fails
	\begin{align}
		\sum_{i \in \mathcal{X}} \pr{X=i} = 2.1
	\end{align}
	
	\end{enumerate}
	
\end{document}