\documentclass[journal,12pt,twocolumn]{IEEEtran}

\usepackage{tfrupee}
\usepackage{enumerate}
\usepackage{amsmath}
\usepackage{amssymb}

\title{Assignment 1 \\ \Large AI1110: Probability and Random Variables \\ \large Indian Institute of Technology Hyderabad}
\author{Ankit Saha \\ \normalsize AI21BTECH11004 \\ \vspace*{20pt} \normalsize  29 March 2022 \\ \vspace*{20pt} \Large ICSE 2019 Grade 10}

\begin{document}
	% The title
	\maketitle
	
	% The question
	\textbf{Question 1(b)} 
	A man invests \rupee~$4500$ in shares of a company which is paying $7.5\%$ dividend. If \rupee~$100$ shares are available at a discount of $10\%$, find:
	\begin{enumerate}[(i)]
		\item the number of shares he purchases
		\item his annual income
	\end{enumerate}
	
	\vspace*{10pt}
	
	% The solution
	\textbf{Solution.}
	
	Total investment made by the man, $P = $ \rupee~$4500$
	
	Face value of a share, $F = $ \rupee~$100$
	
	Discount on shares, $d = 10\%$
	
	Dividend, $D = 7.5\%$
	
	\vspace*{5pt}	
	
	\begin{enumerate}[(i)]
		\item Market value of a share, $M = F \left(1 - \dfrac{d}{100} \right)$
		\vspace*{5pt}
		
		The number of shares purchased is given by: 		
		\begin{align*}
			N &= \dfrac{P}{M} \\ 
			&= \dfrac{P}{F \left(1 - \dfrac{d}{100} \right)} \\
			&= \dfrac{P}{F \left(\dfrac{100 - d}{100} \right)} \\
 			\therefore~N &= \dfrac{100P}{F(100-d)}
		\end{align*}
		
		On substituting the values, we get:
		\begin{align*}
			N = \dfrac{100 \times 4500}{100(100 - 10)} = \dfrac{4500}{90} = 50
		\end{align*}
		
		$\therefore$ The man purchased $50$ shares.
		
		\vspace*{10pt}
		\item His annual income is given by:
		\begin{align*}
		 	A &= F \times N \times \dfrac{D}{100} \\ 
		 	&= F \times \dfrac{100P}{F(100-d)} \times \dfrac{D}{100} \\ 
		 	\therefore~A &= \dfrac{PD}{100-d}
		 \end{align*}
		 
		 On substituting the values, we get:
		\begin{align*}
			A = \dfrac{4500 \times 7.5}{100 - 10} = \dfrac{4500 \times 7.5}{90} = 50 \times 7.5 = 375
		\end{align*}
		
		$\therefore$ The annual income of the man is \rupee~$375$		
		
	\end{enumerate}
	
\end{document}