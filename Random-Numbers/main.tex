\documentclass[journal,12pt,twocolumn]{IEEEtran}

\usepackage{enumitem}
\usepackage{amsmath}
\usepackage{amssymb}
\usepackage{gensymb}
\usepackage{graphicx}
\usepackage{txfonts}         
\usepackage{listings}
\usepackage{lstautogobble}
\usepackage{mathtools}

\newcommand{\solution}{\noindent \textbf{Solution: }}
\providecommand{\pr}[1]{\ensuremath{\Pr\left(#1\right)}}
\providecommand{\brak}[1]{\ensuremath{\left(#1\right)}}
\providecommand{\cbrak}[1]{\ensuremath{\left\{#1\right\}}}
\providecommand{\sbrak}[1]{\ensuremath{\left[#1\right]}}
\providecommand{\mean}[1]{E\left[ #1 \right]}
\providecommand{\var}[1]{\mathrm{Var}\left[ #1 \right]}
\providecommand{\der}[1]{\mathrm{d} #1}
\providecommand{\gauss}[2]{\mathcal{N}\ensuremath{\left(#1,#2\right)}}
\providecommand{\mbf}{\mathbf}

\let\StandardTheFigure\thefigure
\let\vec\mathbf

\numberwithin{equation}{section}
\renewcommand{\thefigure}{\theenumi}
\renewcommand\thesection{\arabic{section}}

\newcommand{\myvec}[1]{\ensuremath{\begin{pmatrix}#1\end{pmatrix}}}
\newcommand{\mydet}[1]{\ensuremath{\begin{vmatrix}#1\end{vmatrix}}}

\lstset {
	frame=single, 
	breaklines=true,
	columns=fullflexible,
	autogobble=true
}             
                               
\title{Random Numbers \\ \Large AI1110: Probability and Random Variables \\ \large Indian Institute of Technology Hyderabad}
\author{Ankit Saha \\ \normalsize AI21BTECH11004 \\ \vspace*{20pt} \normalsize  30 Jun 2022}


\begin{document}

	\maketitle
	
	\section{Uniform Random Numbers}
	Let $U$ be a uniform random variable between 0 and 1.
	\begin{enumerate}[label=\thesection.\arabic*,ref=\thesection.\theenumi]
	\item Generate $10^6$ samples of $U$ using a C program and save into a file called uni.dat

	\solution Download the C source code by executing the following commands
	\begin{lstlisting}
		wget https://github.com/Ankit-Saha-2003/AI1110/raw/main/Random-Numbers/codes/1.1.c
		wget https://github.com/Ankit-Saha-2003/AI1110/raw/main/Random-Numbers/codes/header.h
	\end{lstlisting}
	Compile and run the C program by executing the following
	\begin{lstlisting}
		cc -lm 1.1.c
		./a.out
	\end{lstlisting}
	
	\item Load the uni.dat file into Python and plot the empirical CDF of $U$ using the samples in uni.dat. The CDF is defined as
	\begin{align}
		F_{U}(x) = \pr{U \le x}
	\end{align}

	\solution  Download the following Python code that plots Fig. \ref{fig-1.2}
	\begin{lstlisting}
		wget https://github.com/Ankit-Saha-2003/AI1110/raw/main/Random-Numbers/codes/1.2.py
	\end{lstlisting}
	Run the code by executing
	\begin{lstlisting}
		python 1.2.py
	\end{lstlisting}
	\begin{figure}
		\centering
		\includegraphics[width=\columnwidth]{./figs/1.2.png}
		\caption{The CDF of $U$}
		\label{fig-1.2}
	\end{figure}
	
	\item Find a  theoretical expression for $F_{U}(x)$
	
	\solution The PDF of $U$ is given by
	\begin{align}
		p_{U}(x) = 
		\begin{cases}
			1 & x \in [0, 1] \\
			0 & \text{otherwise}
		\end{cases}
	\end{align}
	
	The CDF of $U$ is given by
	\begin{align}
		F_{U}(x) = \pr{U \le x} = \int_{-\infty}^x p_{U}(x) ~\mathrm{d}x
	\end{align}
	
	If $x<0$,
	\begin{align}
		\int_{-\infty}^x p_{U}(x) ~\mathrm{d}x = \int_{-\infty}^x 0 ~\mathrm{d}x = 0
	\end{align}
	
	If $c$,
	\begin{align}
		\int_{-\infty}^x p_{U}(x) ~\mathrm{d}x &= \int_{-\infty}^0 0 ~\mathrm{d}x + \int_0^x 1 ~\mathrm{d}x \\
		&= 0 + x \\
		&= x
	\end{align}
	
	If $x>1$,
	\begin{multline}
		\int_{-\infty}^x p_{U}(x) ~\mathrm{d}x \\= \int_{-\infty}^0 0 ~\mathrm{d}x + \int_0^1 1 ~\mathrm{d}x +  \int_1^x 0 ~\mathrm{d}x 
	\end{multline}
	\begin{align}
		\int_{-\infty}^x p_{U}(x) ~\mathrm{d}x &= 0 + 1 + 0 \\
		&= 1
	\end{align}
	
	Therefore, we obtain the CDF of $U$ as
	\begin{align}
		F_{U}(x) = 
		\begin{cases}
			0 & x < 0 \\
			x & 0 \le x \le 1 \\
			1 & x > 1
		\end{cases}
	\end{align}
	
	\item The mean of $U$ is defined as
	\begin{align}
		\mean{U} = \frac{1}{N}\sum_{i=1}^{N}U_i
	\end{align}
	and its variance as
	\begin{align}
		\var{U} = \mean{U- \mean{U}}^2 
	\end{align}
	Write a C program to  find the mean and variance of $U$
	
	\solution Download the C source code by executing the following commands
	\begin{lstlisting}
		wget https://github.com/Ankit-Saha-2003/AI1110/raw/main/Random-Numbers/codes/1.4.c
		wget https://github.com/Ankit-Saha-2003/AI1110/raw/main/Random-Numbers/codes/header.h
	\end{lstlisting}
	Compile and run the C program by executing the following
	\begin{lstlisting}
		cc -lm 1.4.c
		./a.out
	\end{lstlisting}
	The output of the code is
	\begin{align}
		\mu_{\text{emp}} &= 0.500007 \\
		\mu_{\text{the}} &= 0.500000 \\
		\sigma_{\text{emp}}^2 &= 0.083301 \\
		\sigma_{\text{the}}^2 &= 0.083333
	\end{align}
	
	\item Verify your result theoretically given that
	\begin{align}
		\mean{U^k} = \int_{-\infty}^{\infty}x^k \mathrm{d}F_{U}(x)
	\end{align}
		
	\solution The mean of $U$ is given by
	\begin{align}
		\mean{U} = \int_{-\infty}^{\infty}x ~\mathrm{d}F_{U}(x) 
	\end{align}
	On differentiating the CDF of $U$, we get
	\begin{align}
		\mathrm{d}F_{U}(x) = 
		\begin{cases}
			0 & x < 0 \\
			\mathrm{d}x & 0 \le x \le 1 \\
			0 & x > 1
		\end{cases}
	\end{align}
	\begin{align}
		\therefore \mean{U} = \int_{0}^{1}x ~\mathrm{d}x = \frac12 = 0.5
	\end{align}
	
	Similarly,
	\begin{align}
		\therefore \mean{U^2} = \int_{0}^{1}x^2 ~\mathrm{d}x = \frac13
	\end{align}
	Now, the variance of $U$ is given by
	\begin{align}
		&\var{U} \\
		&= \mean{U- \mean{U}}^2 \\
		&= \mean{U^2 - 2U\mean{U} + (\mean{U})^2}
	\end{align}
	By linearity of expectation, we have
	\begin{align}
		&\mean{U^2} + \mean{-2U\mean{U}} + \mean{(\mean{U})^2} \\
		&= \mean{U^2} -2\mean{U}\mean{U} + (\mean{U})^2 \\
		&= \mean{U^2} - (\mean{U})^2 \\
		&= \frac13 - \brak{\frac12}^2 \\
		&= \frac{1}{12} \approx 0.083333
	\end{align}
	
	\end{enumerate}
	
	\section{Central Limit Theorem}

	\begin{enumerate}[label=\thesection.\arabic*,ref=\thesection.\theenumi]
	\item Generate $10^6$ samples of the random variable
	\begin{align}
		X = \sum_{i=1}^{12}U_i -6
	\end{align}

	using a C program, where $U_i, i = 1,2,\dots, 12$ are  a set of independent uniform random variables between 0 and 1 and save in a file called gau.dat
	
	\solution Download the C source code by executing the following commands
	\begin{lstlisting}
		wget https://github.com/Ankit-Saha-2003/AI1110/raw/main/Random-Numbers/codes/2.1.c
		wget https://github.com/Ankit-Saha-2003/AI1110/raw/main/Random-Numbers/codes/header.h
	\end{lstlisting}
	Compile and run the C program by executing the following
	\begin{lstlisting}
		cc -lm 2.1.c
		./a.out
	\end{lstlisting}
		
	\item Load gau.dat in Python and plot the empirical CDF of $X$ using the samples in gau.dat. What properties does a CDF have?

	\solution Download the following Python code that plots Fig. \ref{fig-2.2}
	\begin{lstlisting}
		wget https://github.com/Ankit-Saha-2003/AI1110/raw/main/Random-Numbers/codes/2.2.py
	\end{lstlisting}
	Run the code by executing
	\begin{lstlisting}
		python 2.2.py
	\end{lstlisting}
	\begin{figure}
		\centering
		\includegraphics[width=\columnwidth]{./figs/2.2.png}
		\caption{The CDF of $X$}
		\label{fig-2.2}
	\end{figure}
	
	Every CDF is monotone increasing and right-continuous. Furthermore,
	\begin{align}
		\lim_{x \to -\infty} F_{X}(x) = 0 \qquad \lim_{x \to \infty} F_{X}(x) = 1
	\end{align}
	Thus, every CDF is bounded between $0$ and $1$ and hence, convergent.
	
	In this case, the CDF is also left-continuous. Therefore, $X$ is a continuous random variable.
	
	\item Load gau.dat in Python and plot the empirical PDF of $X$ using the samples in gau.dat. The PDF of $X$ is defined as
	\begin{align}
		p_{X}(x) = \frac{\der{}}{\der{x}}F_{X}(x)
	\end{align}
	What properties does the PDF have?
	
	\solution Download the following Python code that plots Fig. \ref{fig-2.2}
	\begin{lstlisting}
		wget https://github.com/Ankit-Saha-2003/AI1110/raw/main/Random-Numbers/codes/2.3.py
	\end{lstlisting}
	Run the code by executing
	\begin{lstlisting}
		python 2.3.py
	\end{lstlisting}
	\begin{figure}
		\centering
		\includegraphics[width=\columnwidth]{./figs/2.3.png}
		\caption{The PDF of $X$}
		\label{fig-2.3}
	\end{figure}
	
	Every PDF is bounded between $0$ and $1$ and
	\begin{align}
		\int_{-\infty}^{\infty} p_{X}(x) ~\mathrm{d}x = 1
	\end{align}
	
	In this case, the PDF is symmetric about $x = 0$
	
	\item Find the mean and variance of $X$ by writing a C program
	
	\solution Download the C source code by executing the following commands
	\begin{lstlisting}
		wget https://github.com/Ankit-Saha-2003/AI1110/raw/main/Random-Numbers/codes/2.4.c
		wget https://github.com/Ankit-Saha-2003/AI1110/raw/main/Random-Numbers/codes/header.h
	\end{lstlisting}
	Compile and run the C program by executing the following
	\begin{lstlisting}
		cc -lm 2.4.c
		./a.out
	\end{lstlisting}
	The output of the code is
	\begin{align}
		\mu_{\text{emp}} &= 0.000294 \\
		\mu_{\text{the}} &= 0.000000 \\
		\sigma_{\text{emp}}^2 &= 0.999560 \\
		\sigma_{\text{the}}^2 &= 1.000000
	\end{align}	
	
	\item Given that 
	\begin{align}
		p_{X}(x) = \frac{1}{\sqrt{2\pi}}\exp\brak{-\frac{x^2}{2}}, -\infty < x < \infty,
	\end{align}
	repeat the above exercise theoretically
	
	\solution The mean of $X$ is given by
	\begin{align}
		\mean{X} &= \int_{-\infty}^{\infty} x p_{X}(x) \mathrm{d}x \\
		&= \int_{-\infty}^{\infty} \frac{x}{\sqrt{2\pi}}\exp\brak{-\frac{x^2}{2}} \mathrm{d}x 
	\end{align}
	Now, let
	\begin{align} 
		g(x) &= \dfrac{x}{\sqrt{2\pi}}\exp\brak{-\frac{x^2}{2}} \\
		\implies g(-x) &= \dfrac{-x}{\sqrt{2\pi}}\exp\brak{-\frac{(-x)^2}{2}} \\
		&= - \dfrac{x}{\sqrt{2\pi}}\exp\brak{-\frac{x^2}{2}} \\
		&= - g(x)
	\end{align}
	Thus, $g(x)$ is an odd function 
	\begin{align}
		\therefore \mean{X} &= \int_{-\infty}^{\infty} g(x) \mathrm{d}x = 0
	\end{align}
	
	Now, 
	\begin{align}
		\mean{X^2} &= \int_{-\infty}^{\infty} x^2 p_{X}(x) \mathrm{d}x \\
		&= \int_{-\infty}^{\infty} \frac{x^2}{\sqrt{2\pi}}\exp\brak{-\frac{x^2}{2}} \mathrm{d}x \\
		&= 2 \int_{0}^{\infty} \frac{x^2}{\sqrt{2\pi}}\exp\brak{-\frac{x^2}{2}} \mathrm{d}x
	\end{align}
	since $\frac{x^2}{\sqrt{2\pi}}\exp\brak{-\frac{x^2}{2}}$ is an even function
	
	Using integration by parts,
	\begin{align}
		\mean{X^2} = \sqrt{\frac{2}{\pi}}  \int_{0}^{\infty} x \cdot x \exp\brak{-\frac{x^2}{2}} \der{x} 
	\end{align}
	\begin{multline}
		= \sqrt{\frac{2}{\pi}} \brak{\left. x \int x \exp\brak{-\frac{x^2}{2}} \der{x}}\right|_0^{\infty} \\- \sqrt{\frac{2}{\pi}}  \int_{0}^{\infty} 1 \cdot \int x \exp\brak{-\frac{x^2}{2}} \der{x}
	\end{multline}
	
	Substitute $t = \frac{x^2}{2} \implies \der{t} = x\der{x}$
	\begin{align}
		\int x \exp\brak{-\frac{x^2}{2}} \der{x} &= \int \exp(-t) \der{t} \\
		&= - \exp(-t) \\
		&= - \exp\brak{-\frac{x^2}{2}}
	\end{align}
	Now,
	\begin{align}
		\left. -x \exp\brak{-\frac{x^2}{2}} \right|_0^{\infty} = 0 - 0 = 0 \\
		\because \lim_{x\to\infty} x \exp\brak{-\frac{x^2}{2}} = \lim_{x\to\infty} \frac{x}{\exp\brak{\frac{x^2}{2}}} =0
	\end{align}
	as exponential function grows much faster than a polynomial function
	
	Also, 
	\begin{align}
		&\int_0^{\infty} - \exp\brak{-\frac{x^2}{2}} \der{x} \\
		\xleftrightarrow{x = t\sqrt{2}} &\int_0^{\infty} -\exp(-t^2) \der{t}\sqrt{2} \\
		&= -{\sqrt{2}} \int_0^{\infty} \exp(-t^2) \der{t} \\
		&= -{\sqrt{2}} \frac{\sqrt{\pi}}{2} \\
		&= - \sqrt{\frac{\pi}{2}}
	\end{align}
	
	Therefore,
	\begin{align}
		\mean{X^2} &= 0 - \sqrt{\frac{2}{\pi}} \brak{- \sqrt{\frac{\pi}{2}}} \\
		&= 1 \\
		\therefore \var{X} &= \mean{X^2} - \brak{\mean{X}}^2 \\
		&= 1 - 0 \\
		&= 1
	\end{align}
	\end{enumerate}
	
	\section{From Uniform to Other}
	\begin{enumerate}[label=\thesection.\arabic*,ref=\thesection.\theenumi]
	\item Generate samples of 
	\begin{align}
		V = -2\ln\brak{1-U}
	\end{align}
	and plot its CDF
	
	\solution Download the C source code by executing the following commands
	\begin{lstlisting}
		wget https://github.com/Ankit-Saha-2003/AI1110/raw/main/Random-Numbers/codes/3.1.c
	\end{lstlisting}
	Compile and run the C program by executing the following
	\begin{lstlisting}
		cc -lm 3.1.c
		./a.out
	\end{lstlisting}
	Download the following Python code that plots Fig. \ref{fig-3.1}
	\begin{lstlisting}
		wget https://github.com/Ankit-Saha-2003/AI1110/raw/main/Random-Numbers/codes/3.1.py
	\end{lstlisting}
	Run the code by executing
	\begin{lstlisting}
		python 3.1.py
	\end{lstlisting}
	\begin{figure}
		\centering
		\includegraphics[width=\columnwidth]{./figs/3.1.png}
		\caption{The CDF of $V$}
		\label{fig-3.1}
	\end{figure}	
	
	\item Find a theoretical expression for $F_V(x)$
	
	\solution We have
	\begin{align}
		F_V(x) &= \pr{V \le x} \\
		&= \pr{-2\ln\brak{1-U} \le x} \\
		&= \pr{\ln\brak{1-U} \ge -\frac{x}{2}} \\
		&= \pr{1-U \ge \exp\brak{-\frac{x}{2}}} \\
		&= \pr{U \le 1 - \exp\brak{-\frac{x}{2}}} \\
		&= F_U\brak{1 - \exp\brak{-\frac{x}{2}}}
	\end{align}
	Now,
	\begin{align}
		0 \le 1-\exp\brak{-\frac{x}{2}} &< 1 \qquad \text{if } x \ge 0	\\	
		1-\exp\brak{-\frac{x}{2}} &< 0 \qquad \text{if } x < 0	
	\end{align}
	
	Therefore,
	\begin{align}
		F_V(x) = 
		\begin{cases}
			1-\exp\brak{-\dfrac{x}{2}} & x \ge 0 \\
			0 & x < 0
		\end{cases}
	\end{align}
	
	\end{enumerate}
	
	\section{Triangular Distribution}
	\begin{enumerate}[label=\thesection.\arabic*,ref=\thesection.\theenumi]
	\item Generate 
	\begin{align}
		T = U_1+U_2
	\end{align}
	\solution Download the C source code by executing the following commands
	\begin{lstlisting}
		wget https://github.com/Ankit-Saha-2003/AI1110/raw/main/Random-Numbers/codes/4.1.c
	\end{lstlisting}
	Compile and run the C program by executing the following
	\begin{lstlisting}
		cc -lm 4.1.c
		./a.out
	\end{lstlisting}
	
	\item Find the CDF of $T$
	
	\solution The CDF of $T$ is given by
	\begin{align}
		F_T(t) = \pr{T \le t} = \pr{U_1 + U_2 \le t}	
	\end{align}		
	Since $U_1, U_2 \in [0,1] \implies U_1 + U_2 \in [0,2]$
	Therefore, if $t \ge 2$, then $U_1 + U_2 \le t$ is always true and if $t < 0$, then $U_1 + U_2 \le t$ is always false.
	
	Now, fix the value of $U_1$ to be some $x$
	\begin{align}
		x + U_2 \le t \implies U_2 \le t - x
	\end{align}
	
	If $0 \le t \le 1$, then $x$ can take all values in $[0,t]$
	\begin{align}
		F_T(t)	&= \int_0^t \pr{U_2 \le t - x} p_{U_1}(x) \der{x} \\
		&= \int_0^t F_{U_2}(t-x) p_{U_1}(x) \der{x}
	\end{align}
	\begin{align}
		0 \le x \le t &\implies 0 \le t - x \le t \le 1 \\
		&\implies F_{U_2}(t-x) = t - x
	\end{align}
	\begin{align}
		F_T(t) &= \int_0^t (t-x) \cdot 1 \cdot \der{x} \\
		&= \left. tx - \frac{x^2}{2} \right|_0^t \\
		&= \frac{t^2}{2}
	\end{align}
	
	If $1 < t < 2$, $x$ can only take values in $[0,1]$ as $U_1 \le 1$
	\begin{align}
		F_T(t)	&= \int_0^1 F_{U_2}(t-x) \cdot 1 \cdot \der{x} 
	\end{align}
	\begin{align}
		0 \le x \le t - 1 &\implies 1 \le t - x \le t \\
		t - 1 \le x \le 1 &\implies 0 < t - 1 \le t - x \le 1
	\end{align}
	\begin{align}
		F_T(t) &= \int_0^{t-1} 1 \der{x} + \int_{t-1}^1 (t-x)\der{x} \\
		&= t - 1 + t(1 - (t - 1)) - \frac{1}{2} + \frac{(t-1)^2}{2} \\
		&= t - 1 + 2t - t^2 -\frac{1}{2} + \frac{t^2}{2} + \frac{1}{2} - t \\ 
		&= -\frac{t^2}{2} + 2t - 1
	\end{align}
	
	Therefore,
	\begin{align}
		F_T(t) = 
		\begin{cases}
		0 & t < 0 \\
		\dfrac{t^2}{2} & 0 \le t \le 1 \\
		 2t -\dfrac{t^2}{2} - 1 & 1 < t < 2 \\
		1 & t \ge 2
		\end{cases}
	\end{align}
	
	\begin{figure}
		\centering
		\includegraphics[width=\columnwidth]{./figs/4.2.png}
		\caption{The CDF of $T$}
		\label{fig-4.2}
	\end{figure}	
	
	\item Find the PDF of $T$
	
	\solution The PDF of $T$ is given by
	\begin{align}
		p_T(t) &= \frac{\der{}}{\der{t}} F_T(t) \\
		\therefore p_T(t) &=
		\begin{cases}
			0 & t < 0 \\
			t & 0 \le t \le 1 \\
			2 - t & 1 < t < 2 \\
			0 & t \ge 2
		\end{cases}
	\end{align}
	
	\begin{figure}
		\centering
		\includegraphics[width=\columnwidth]{./figs/4.3.png}
		\caption{The PDF of $T$}
		\label{fig-4.3}
	\end{figure}		
	
	\item Find the theoretical expressions for the PDF and CDF of $T$
	
	\solution The theoretical expressions for the CDF and PDF have been found in problems 4.2 and 4.3 respectively

	\item Verify your results through a plot
	
	\solution Download the following Python codes that plot Fig. \ref{fig-4.2} and Fig. \ref{fig-4.3}
	\begin{lstlisting}
		wget https://github.com/Ankit-Saha-2003/AI1110/raw/main/Random-Numbers/codes/4.2.py
		wget https://github.com/Ankit-Saha-2003/AI1110/raw/main/Random-Numbers/codes/4.3.py
	\end{lstlisting}
	Run the codes by executing
	\begin{lstlisting}
		python 4.2.py
		python 4.3.py
	\end{lstlisting}
	
	
	\end{enumerate}
	
	\section{Maximum Likelihood}
	\begin{enumerate}[label=\thesection.\arabic*,ref=\thesection.\theenumi]
	\item Generate equiprobable 	$X \in \cbrak{-1, 1}$
	
	\solution Download the C source code by executing the following commands
	\begin{lstlisting}
		wget https://github.com/Ankit-Saha-2003/AI1110/raw/main/Random-Numbers/codes/5.1.c
		wget https://github.com/Ankit-Saha-2003/AI1110/raw/main/Random-Numbers/codes/header.h
	\end{lstlisting}
	Compile and run the C program by executing the following
	\begin{lstlisting}
		cc -lm 5.1.c
		./a.out
	\end{lstlisting}
	
	\item Generate 
	\begin{align}
		Y = AX+N
	\end{align}
	where $A = 5 ~\mathrm{dB}, X \in \cbrak{-1, 1}$ is Bernoulli and $N \sim \gauss{0}{1}$
	
	\solution Download the C source code by executing the following commands
	\begin{lstlisting}
		wget https://github.com/Ankit-Saha-2003/AI1110/raw/main/Random-Numbers/codes/5.2.c
		wget https://github.com/Ankit-Saha-2003/AI1110/raw/main/Random-Numbers/codes/header.h
	\end{lstlisting}
	Compile and run the C program by executing the following
	\begin{lstlisting}
		cc -lm 5.2.c
		./a.out
	\end{lstlisting}
	
	\item Plot $Y$
	
	\solution  Download the following Python code that plots Fig. \ref{fig-5.3}
	\begin{lstlisting}
		wget https://github.com/Ankit-Saha-2003/AI1110/raw/main/Random-Numbers/codes/5.3.py
	\end{lstlisting}
	Run the code by executing
	\begin{lstlisting}
		python 5.3.py
	\end{lstlisting}
	\begin{figure}
		\centering
		\includegraphics[width=\columnwidth]{./figs/5.3.png}
		\caption{Plot of $Y$}
		\label{fig-5.3}
	\end{figure}
	
	\item Guess how to estimate $X$ from $Y$
	
	\solution
	\begin{align}
		\hat{X} = 
		\begin{cases}
			1 & Y > 0 \\
			-1 & Y < 0
		\end{cases}
	\end{align}
	
	\item Find 
	\begin{align}
		P_{e|0} = \pr{\hat{X} = -1|X=1}
	\end{align}
	and 
	\begin{align}
		P_{e|1} = \pr{\hat{X} = 1|X=-1}
	\end{align}
	
	\solution 
	\begin{align}
		\pr{\hat{X}=-1|X=1} &= \pr{Y<0|X=1} \\
		&= \pr{A+N<0} \\
		&= \pr{N<-A} \\
		&= 1 - \pr{N>-A} \\
		&= 1 - Q(-A)	\\
		&= Q(A)
	\end{align}
	where $Q(x) = \pr{N>x}$ is the Q-function	
	\begin{align}
		Q(x) = 1 - Q(-x) \qquad \forall x \in \mathbb{R}
	\end{align}
	
	\begin{align}
		\pr{\hat{X}=1|X=-1} &= \pr{Y>0|X=-1} \\
		&= \pr{-A+N>0} \\
		&= \pr{N>A} \\
		&= Q(A)		
	\end{align}
	
	\item Find $P_e$ assuming that $X$ has equiprobable symbols
	
	\solution 
	\begin{align}
		P_e = \pr{X=-1} P_{e|0} + \pr{X=1} P_{e|1}
	\end{align}
	Since $X$ has equiprobable symbols, $\pr{X=-1} = \pr{X=1} = \frac12$
	\begin{align}
		P_e &= \frac12 Q(A) + \frac12 Q(A) \\
		&= Q(A)	
	\end{align}		
	
	\item Verify by plotting the theoretical $P_e$ with respect to $A$ from $0$ to $10 ~\mathrm{dB}$
	
	\solution Download the following Python code that plots Fig. \ref{fig-5.7}
	\begin{lstlisting}
		wget https://github.com/Ankit-Saha-2003/AI1110/raw/main/Random-Numbers/codes/5.7.py
	\end{lstlisting}
	Run the code by executing
	\begin{lstlisting}
		python 5.7.py
	\end{lstlisting}
	\begin{figure}
		\centering
		\includegraphics[width=\columnwidth]{./figs/5.7.png}
		\caption{Plot of $P_e$}
		\label{fig-5.7}
	\end{figure}
	
	\item Now, consider a threshold $\delta$  while estimating $X$ from $Y$. Find the value of $\delta$ that minimizes the theoretical $P_e$
	
	\solution 
	\begin{align}
		X &= 
		\begin{cases}
			1 & Y > \delta \\
			-1 & Y < \delta
		\end{cases} \\
		P_{e|0} &= \pr{\hat{X} = -1|X=1} \\
		&= \pr{Y<\delta | X=1} \\
		&= \pr{A+N < \delta} \\
		&= \pr{N < \delta - A} \\
		&= Q(A - \delta) \\
		P_{e|1} &= \pr{\hat{X} = 1|X=-1} \\
		&= \pr{Y>\delta | X=-1} \\
		&= \pr{-A+N > \delta} \\
		&= \pr{N > \delta + A} \\
		&= Q(A + \delta) \\
	\end{align}
	Now, $P_e$ is given by
	\begin{align}
		P_e &= \pr{X=-1} P_{e|0} + \pr{X=1} P_{e|1} \\
		&= \frac12 Q(A - \delta) + \frac12 Q(A + \delta) \\
		&= \frac{Q(A - \delta) + Q(A + \delta)}{2} \\
		&= g(\delta) 
	\end{align}
	
	On differentiating $g$ with respect to $\delta$, we get
	\begin{align}
		g'(\delta) = \frac{Q'(A + \delta) - Q'(A - \delta)}{2}
	\end{align}
	
	Recall the definition of $Q(x)$
	\begin{align}
		Q(x) &= \int_x^\infty \frac{1}{\sqrt{2\pi}} \exp\brak{-\frac{u^2}{2}} \der{u} \\
		\implies Q'(x) &= - \frac{1}{\sqrt{2\pi}} \exp\brak{-\frac{x^2}{2}}
	\end{align}
	
	Thus,
	\begin{align}
		g'(\delta) &= \frac{\exp\brak{-\frac{(A-\delta)^2}{2}} - \exp\brak{-\frac{(A+\delta)^2}{2}}}{2\sqrt{2\pi}} \\
		g'(\delta) = 0 &\implies (A-\delta)^2 = (A+\delta)^2 \\
		&\implies |A-\delta| = |A+\delta| \\
		&\implies \delta = 0
	\end{align}
	\begin{multline}
		g''(\delta) = \frac{(A-\delta)}{2\sqrt{2\pi}} \exp\brak{-\frac{(A-\delta)^2}{2}} \\+ \frac{(A+\delta)}{2\sqrt{2\pi}} \exp\brak{-\frac{(A+\delta)^2}{2}}
	\end{multline}
	\begin{align}
		g''(0) = \frac{A}{\sqrt{2\pi}} \exp\brak{-\frac{A^2}{2}} > 0 \quad (\because A>0)
	\end{align}
	
	Therefore, $\hat{\delta} = 0$ is a minima and it is what minimizes $P_e$
	
	\item Repeat the above exercise when 
	\begin{align}
		p_{X}(-1) = p
	\end{align}
	
	\solution	
	\begin{align}
		P_e &= p_X(1) P_{e|0} + p_X(-1) P_{e|1} \\
		&= (1-p) Q(A-\delta) + p Q(A+\delta) \\
		&= g(\delta)
	\end{align}
	
	On differentiating $g$ with respect to $\delta$, we get
	\begin{align}
		g'(\delta) = \frac{(1-p)\exp\brak{-\frac{(A-\delta)^2}{2}} - p\exp\brak{-\frac{(A+\delta)^2}{2}}}{\sqrt{2\pi}} 
	\end{align}
	
	$g'(\delta) = 0$ when
	\begin{align}
			&(1-p)\exp\brak{-\frac{(A-\delta)^2}{2}} = p\exp\brak{-\frac{(A+\delta)^2}{2}} \\
			\implies &\exp\brak{\frac{(A+\delta)^2 - (A-\delta)^2}{2}} = \frac{p}{1-p} \\
			\implies &\exp\brak{2A\delta} = \frac{p}{1-p} \\
			\therefore &~\hat{\delta} = \frac{1}{2A} \ln\frac{p}{1-p}
	\end{align}
	
	\item Repeat the above exercise using the MAP criterion	
	
	\solution The PDF of $X|Y$ is given by
	\begin{align}
		p_{X|Y}(x|y) = 	\frac{p_{Y|X}(y|x) p_X(x)}{p_Y(y)}
	\end{align}
		 
	Assuming $X$ has equiprobable symbols, $p_X(x) = \frac12, \quad x = -1, 1$
	
	\begin{align}
		p_Y(y) &= p_X(-1) p_{Y|X}(y|-1) + p_X(1) p_{Y|X}(y|1) \\
		&= \frac12 \pr{-A+N=y} + \frac12 \pr{A+N=y} \\
		&= \frac{p_N(y+A) + p_N(y-A)}{2} \\
		&= \frac{\exp\brak{-\frac{(y+A)^2}{2}} + \exp\brak{-\frac{(y-A)^2}{2}}}{2\sqrt{2\pi}} \\
	\end{align}
	
	Now,
	\begin{align}		
		p_{X|Y}(1|y) &= \frac{\pr{A+N=y} p_X(1)}{p_Y(y)} \\
		&= \frac{p_N(y-A) p_X(1)}{p_Y(y)} \\
		&= \frac{\exp\brak{-\frac{(y-A)^2}{2}}}{2\brak{\exp\brak{-\frac{(y+A)^2}{2}} + \exp\brak{-\frac{(y-A)^2}{2}}}} \\
		&= \frac{1}{2\brak{1 + \exp\brak{\frac{(y-A)^2-(y+A)^2}{2}}}} \\
		&= \frac{1}{2\brak{1 + \exp(-2Ay)}}
	\end{align}
	
	Similarly,
	\begin{align}
		p_{X|Y}(-1|y) &= \frac{1}{2\brak{1 + \exp(2Ay)}}
	\end{align}
	
	Now,
	\begin{align}
		&p_{X|Y}(1|y) > p_{X|Y}(-1|y) \\
		\iff &\frac{1}{2\brak{1 + \exp(-2Ay)}} > \frac{1}{2\brak{1 + \exp(2Ay)}} \\
		\iff &\exp(-2Ay) < \exp(2Ay) \\
		\iff &y > 0
	\end{align}
	And $p_{X|Y}(1|y) < p_{X|Y}(-1|y) \iff y < 0$
	
	Therefore, $X=1$ is more probable than $X=-1$ when $Y>0$ and vice versa
	
	Consider now a general Bernoulli random variable $X$ with $p_X(-1) = p, p_X(1) = 1-p$
	\begin{align}
		p_Y(y) &= p_X(-1) p_{Y|X}(y|-1) + p_X(1) p_{Y|X}(y|1) \\
		&= p p_N(y+A) + (1-p) p_N(y-A) \\
		&= \frac{p\exp\brak{-\frac{(y+A)^2}{2}} + (1-p)\exp\brak{-\frac{(y-A)^2}{2}}}{\sqrt{2\pi}}
	\end{align}
	\begin{align}
		p_{X|Y}(1|y) &= \frac{p_{Y|X}(y|1) p_X(1)}{p_Y(y)} \\
		&= \frac{(1-p)\exp\brak{-\frac{(y-A)^2}{2}}}{p\exp\brak{-\frac{(y+A)^2}{2}} + (1-p)\exp\brak{-\frac{(y-A)^2}{2}}} \\
		&= \frac{1-p}{1-p + p\exp(-2Ay)}
	\end{align}
	
	Similarly,
	\begin{align}
		p_{X|Y}(-1|y) &= \frac{p}{p + (1-p)\exp(2Ay)}
	\end{align}
	
	Now,
	\begin{align}
		&p_{X|Y}(1|y) > p_{X|Y}(-1|y) \\
		&\iff \frac{1-p}{1-p + p\exp(-2Ay)} > \frac{p}{p + (1-p)\exp(2Ay)} \\
		&\iff 1 + \frac{p}{1-p}\exp(-2Ay) < 1 + \frac{1-p}{p}\exp(2Ay) \\
		&\iff \exp(4Ay) > \brak{\frac{p}{1-p}}^2 \\
		&\iff y > \frac{1}{2A} \ln\frac{p}{1-p} = \hat{\delta}
	\end{align}
	
	and
	\begin{align}
		&p_{X|Y}(1|y) < p_{X|Y}(-1|y) \\
		&\iff \frac{1-p}{1-p + p\exp(-2Ay)} < \frac{p}{p + (1-p)\exp(2Ay)} \\
		&\iff 1 + \frac{p}{1-p}\exp(-2Ay) > 1 + \frac{1-p}{p}\exp(2Ay) \\
		&\iff \exp(4Ay) < \brak{\frac{p}{1-p}}^2 \\
		&\iff y < \frac{1}{2A} \ln\frac{p}{1-p} = \hat{\delta}
	\end{align}
	Therefore, $X=1$ is more probable than $X=-1$ when $Y>\hat{\delta}$ and vice versa
	\end{enumerate}
	
	\section{Gaussian to Other}
	\begin{enumerate}[label=\thesection.\arabic*,ref=\thesection.\theenumi]
	\item Let $X_1 \sim  \gauss{0}{1}$ and $X_2 \sim  \gauss{0}{1}$. Plot the CDF and PDF of
	\begin{align}
		V = X_1^2 + X_2^2
	\end{align}
	
	\solution Download the C source code to generate the data by executing the following commands
	\begin{lstlisting}
		wget https://github.com/Ankit-Saha-2003/AI1110/raw/main/Random-Numbers/codes/6.1.c
		wget https://github.com/Ankit-Saha-2003/AI1110/raw/main/Random-Numbers/codes/header.h
	\end{lstlisting}
	Compile and run the C program by executing the following
	\begin{lstlisting}
		cc -lm 6.1.c
		./a.out
	\end{lstlisting}
	
	Download the following Python code that plots Fig. \ref{fig-6.1} 
	\begin{lstlisting}
		wget https://github.com/Ankit-Saha-2003/AI1110/raw/main/Random-Numbers/codes/6.1.py
	\end{lstlisting}
	Run the code by executing
	\begin{lstlisting}
		python 6.1.py
	\end{lstlisting}
	\begin{figure}
		\centering
		\includegraphics[width=\columnwidth]{./figs/6.1.png}
		\caption{CDF of $V$}
		\label{fig-6.1}
	\end{figure}
	\begin{figure}
		\centering
		\includegraphics[width=\columnwidth]{./figs/6.2.png}
		\caption{PDF of $V$}
		\label{fig-6.2}
	\end{figure}
	
	\item If
	\begin{align}
		F_{V}(x) = 
		\begin{cases}
			1 - e^{-\alpha x} & x \geq 0 \\
			0 & x < 0,
		\end{cases}	
	\end{align}
	find $\alpha$
	
	\solution Let $R \ge 0, \Theta \in [0, 2\pi]$
	\begin{align}
		X_1 = R\cos\Theta  \\
		X_2 = R\sin\Theta 
	\end{align}
	such that $V = X_1^2 + X_2^2 = R^2$
	 
	 The Jacobian matrix transforming $R, \Theta$ to $X_1, X_2$  is defined as 
	 \begin{align}
		\vec{J} &= \myvec{
			\frac{\partial X_1}{\partial R} & \frac{\partial X_1}{\partial \Theta} \\
			\frac{\partial X_2}{\partial R} & \frac{\partial X_2}{\partial \Theta}
		} \\
		&= \myvec{
			\cos\Theta & -R\sin\Theta \\
			\sin\Theta & R\cos\Theta
		} \\
		\implies \mydet{\vec{J}} &= R\cos^2\Theta + R\sin^2\Theta = R
	\end{align}
	
	Then,
	\begin{align}
		p_{R, \Theta} \brak{r, \theta} &= p_{X_1,X_2}\brak{x_1,x_2} \mydet{\vec{J}} \\
		&= p_{X_1,X_2}\brak{x_1,x_2} r \\
		&= r p_{X_1}(x_1) p_{X_2}(x_2)
	\end{align}
	since $X_1$ and $X_2$ are independent 
	
	\begin{align}
		p_{R, \Theta} \brak{r, \theta} &= r \frac{1}{\sqrt{2\pi}} \exp\brak{-\frac{x_1^2}{2}} \frac{1}{\sqrt{2\pi}} \exp\brak{-\frac{x_2^2}{2}} \\
		&= \frac{r}{2\pi} \exp\brak{-\frac{(x_1+x_2)^2}{2}} \\
		&= \frac{r}{2\pi} \exp\brak{-\frac{r^2}{2}} 
	\end{align}
	
	The marginal distribution of $R$ is given by
	\begin{align}
		p_R(r) &= \int_0^{2\pi} p_{R, \Theta} \brak{r, \theta} \der{\theta} \\
		&= \frac{r}{2\pi} \exp\brak{-\frac{r^2}{2}} \int_0^{2\pi}\der{\theta} \\
		&= r \exp\brak{-\frac{r^2}{2}} \quad \text{for } r \ge 0
	\end{align}		
	
	The CDF of $R$ is thus given by
	\begin{align}
		F_R(r) &= \pr{R \le r} \\
		&= \int_0^r p_R(r) \der{r} \\
		&= \int_0^r r \exp\brak{-\frac{r^2}{2}} \der{r} \\
		&= - \int_0^r \exp\brak{-\frac{r^2}{2}} (-r\der{r}) \\
		&= - \int_0^r \exp\brak{-\frac{r^2}{2}} \der\brak{-\frac{r^2}{2}} \\
		&= - \left. \exp\brak{-\frac{r^2}{2}} \right|_0^r \\
		&= 1 - \exp\brak{-\frac{r^2}{2}} \quad \text{for } r \ge 0
	\end{align}		
	
	Now, $V = R^2$, hence the CDF of $V$ is given by
	\begin{align}
		F_V(x) &= \pr{V \le x} \\
		&= \pr{R^2 \le x} \\
		&= \pr{|R| \le \sqrt{x}} \\
		&= \pr{R \le \sqrt{x}} && (\because R \ge 0) \\
		&= F_R(\sqrt{x}) \\
		&= 1 - \exp\brak{-\frac{x}{2}} \quad \text{for } x \ge 0
	\end{align}
	
	And the PDF of $V$ is given by
	\begin{align}
		p_V(x) &= \frac{\der{}}{\der{x}} F_V(x) \\
		&= \frac{\der{}}{\der{x}} \brak{1 - \exp\brak{-\frac{x}{2}}} \\
		&= \frac12 \exp\brak{-\frac{x}{2}} 
	\end{align}
	
	Therefore,
	\begin{align}
		F_V(x) &= 
		\begin{cases}
			1 - \exp\brak{-\dfrac{x}{2}} & x \geq 0 \\
			0 & \text{otherwise}
		\end{cases}	\\
		p_V(x) &= 
		\begin{cases}
			\dfrac12 \exp\brak{-\dfrac{x}{2}} & x \geq 0 \\
			0 & \text{otherwise}
		\end{cases}
	\end{align}
	\begin{align}
		\therefore \alpha = \frac12
	\end{align}
	
	\item Plot the CDF and PDF of
	\begin{align}
		A = \sqrt{V}
	\end{align}
	
	\solution Download the C source code to generate the data by executing the following commands
	\begin{lstlisting}
		wget https://github.com/Ankit-Saha-2003/AI1110/raw/main/Random-Numbers/codes/6.3.c
		wget https://github.com/Ankit-Saha-2003/AI1110/raw/main/Random-Numbers/codes/header.h
	\end{lstlisting}
	Compile and run the C program by executing the following
	\begin{lstlisting}
		cc -lm 6.3.c
		./a.out
	\end{lstlisting}
	
	Download the following Python code that plots Fig. \ref{fig-6.3} 
	\begin{lstlisting}
		wget https://github.com/Ankit-Saha-2003/AI1110/raw/main/Random-Numbers/codes/6.3.py
	\end{lstlisting}
	Run the code by executing
	\begin{lstlisting}
		python 6.3.py
	\end{lstlisting}
	\begin{figure}
		\centering
		\includegraphics[width=\columnwidth]{./figs/6.3.png}
		\caption{CDF of $A$}
		\label{fig-6.3}
	\end{figure}
	\begin{figure}
		\centering
		\includegraphics[width=\columnwidth]{./figs/6.4.png}
		\caption{PDF of $A$}
		\label{fig-6.4}
	\end{figure}
	
	The CDF of $A$ for $x \ge 0$ is given by
	\begin{align}
		F_A(x) &= \pr{A \le x} \\
		&= \pr{\sqrt{V} \le x} \\
		&= \pr{V \le x^2} \\
		&= F_V(x^2) \\
		&= 1 - \exp\brak{-\frac{x^2}{2}}
	\end{align}
	
	The PDF of $A$ is given by
	\begin{align}
		p_A(x) &= \frac{\der{}}{\der{x}} F_A(x) \\
		&= \frac{\der{}}{\der{x}} \brak{1 - \exp\brak{-\frac{x^2}{2}}} \\
		&= x \exp\brak{-\frac{x^2}{2}}
	\end{align}
	
	Therefore,
	\begin{align}
		F_A(x) &= 
		\begin{cases}
			1 - \exp\brak{-\dfrac{x^2}{2}} & x \geq 0 \\
			0 & \text{otherwise}
		\end{cases}	\\
		p_A(x) &= 
		\begin{cases}
			x \exp\brak{-\dfrac{x^2}{2}} & x \geq 0 \\
			0 & \text{otherwise}
		\end{cases}
	\end{align}
	
	\end{enumerate}
	
	\section{Conditional Probability}
	\begin{enumerate}[label=\thesection.\arabic*,ref=\thesection.\theenumi]
	\item Plot 
	\begin{align}
		P_e = \pr{\hat{X} = -1|X=1}	
	\end{align}
	for 
	\begin{align}
		Y = AX+N
	\end{align}
	where $A$ is Rayleigh with $E\sbrak{A^2} = \gamma, N \sim \gauss{0}{1}, X \in \cbrak{-1,1}$ for $0 \le \gamma \le 10$ dB
	
	\solution Download the C source code to generate the data by executing the following commands
	\begin{lstlisting}
		wget https://github.com/Ankit-Saha-2003/AI1110/raw/main/Random-Numbers/codes/7.1.c
		wget https://github.com/Ankit-Saha-2003/AI1110/raw/main/Random-Numbers/codes/header.h
	\end{lstlisting}
	Compile and run the C program by executing the following
	\begin{lstlisting}
		cc -lm 7.1.c
		./a.out
	\end{lstlisting}
	
	Download the following Python code that plots Fig. \ref{fig-7.4} 
	\begin{lstlisting}
		wget https://github.com/Ankit-Saha-2003/AI1110/raw/main/Random-Numbers/codes/7.1.py
	\end{lstlisting}
	Run the code by executing
	\begin{lstlisting}
		python 7.1.py
	\end{lstlisting}
	
	\item Assuming that $N$ is a constant, find an expression for $P_e$.  Call this $P_e(N)$
	
	\solution 
	\begin{align}
		P_e(N) &= \pr{\hat{X} = -1|X = 1} \\
		&= \pr{Y<0|X=1} \\
		&= \pr{A+N<0} \\
		&= \pr{A<-N} \\
		&= F_A(-N)
	\end{align}
	
	For Gaussian random variables with $\sigma = 1$, we found the CDF of $A$ to be
	\begin{align}
		F_A(x) &= 
		\begin{cases}
			1 - \exp\brak{-\dfrac{x^2}{2}} & x \geq 0 \\
			0 & \text{otherwise}
		\end{cases}
	\end{align}
	
	For Gaussian random variable $Y$ with a general $\sigma$,
	\begin{align}
		Y &= \sigma X \\
		\implies X &= \frac{Y}{\sigma} 
	\end{align}
	\begin{align}
		F_A(x) &= 
		\begin{cases}
			1 - \exp\brak{-\dfrac{x^2}{2\sigma^2}} & x \geq 0 \\
			0 & \text{otherwise}
		\end{cases} \\
		\implies p_A(x) &=
		\begin{cases}
			\dfrac{x}{\sigma^2}\exp\brak{-\dfrac{x^2}{2\sigma^2}} & x \geq 0 \\
			0 & \text{otherwise}
		\end{cases} \\
		\mean{A^2} &= \int_0^{\infty} x^2 \dfrac{x}{\sigma^2}\exp\brak{-\dfrac{x^2}{2\sigma^2}} \der{x}
	\end{align}
	Substitute $-\dfrac{x^2}{2\sigma^2} = u \implies \der{u} = -\dfrac{x}{\sigma^2} \der{x}$ 
	\begin{align}
		\mean{A^2} &= \int_0^{-\infty} 2\sigma^2u \exp(u) \der{u} \\
		&= 2\sigma^2 \left.\brak{u\exp(u) - \exp(u)}\right|_0^{-\infty} \\
		&= 2\sigma^2 \\
		&= \gamma
	\end{align}
	
	Therefore,
	\begin{align}
		P_e(N) &= 
		\begin{cases}
			1 - \exp\brak{-\dfrac{N^2}{\gamma}} & N \leq 0 \\
			0 & \text{otherwise}
		\end{cases}
	\end{align}
	
	\item For a function $g$,
	\begin{align}
		E\sbrak{g(X)} = \int_{-\infty}^{\infty}g(x)p_{X}(x) \der{x}
	\end{align}
	Find $P_e = E\sbrak{P_e(N)}$
	
	\solution 
	\begin{align}
		P_e &= \int_{-\infty}^{0} P_e(x) p_{N}(x) \der{x} \\
		&= \int_{-\infty}^{0} \brak{1 - \exp\brak{-\dfrac{x^2}{\gamma}}} \frac{1}{\sqrt{2\pi}} \exp\brak{-\frac{x^2}{2}} \der{x} 
	\end{align}
	
	\begin{multline}
		P_e = \frac{1}{\sqrt{2\pi}} \int_{-\infty}^{0} \exp\brak{-\frac{x^2}{2}} \der{x} \\
		- \frac{1}{\sqrt{2\pi}} \int_{-\infty}^{0}  \exp\brak{-x^2\brak{\frac{1}{2} + \frac{1}{\gamma}}} \der{x}
	\end{multline}
	
	\begin{multline}
		P_e = \frac{1}{\sqrt{2\pi}} \int_{-\infty}^{0} \exp\brak{-\frac{x^2}{2}} \der{x} \\
		- \frac{1}{\sqrt{2\pi}} \int_{-\infty}^{0}  \exp\brak{-\frac{(2+\gamma) x^2}{2\gamma} } \der{x}
	\end{multline}
	
	Now,
	\begin{align}
		\int_{-\infty}^{0} \exp\brak{-\frac{x^2}{2a^2}} \der{x} &= \int_{0}^{\infty} \exp\brak{-\frac{x^2}{2a^2}} \der{x} \\
		 &= a \sqrt{\frac{\pi}{2}}
	\end{align}
	
	Therefore,
	\begin{align}
		P_e &= \frac{1}{\sqrt{2\pi}} \sqrt{\frac{\pi}{2}} - \frac{1}{\sqrt{2\pi}} \sqrt{\frac{\pi}{2}} \sqrt{\frac{\gamma}{\gamma+2}} \\
		&= \frac{1}{2} - \frac{1}{2} \sqrt{\frac{\gamma}{\gamma+2}}
	\end{align}
	
	\item Plot $P_e$ in problems 7.1 and 7.3 on the same graph w.r.t $\gamma$.  Comment.
	
	\solution Download the following Python code that plots Fig. \ref{fig-7.4} 
	\begin{lstlisting}
		wget https://github.com/Ankit-Saha-2003/AI1110/raw/main/Random-Numbers/codes/7.1.py
	\end{lstlisting}
	Run the code by executing
	\begin{lstlisting}
		python 7.1.py
	\end{lstlisting}
	
	The graphs in both cases coincide and thus
	\begin{align}
		\pr{\hat{X} = -1|X = 1} = \mean{P_e(N)}
	\end{align}
	
	\begin{figure}
		\centering
		\includegraphics[width=\columnwidth]{./figs/7.1.png}
		\caption{Plot of $P_e$}
		\label{fig-7.4}
	\end{figure}
	\end{enumerate}
	
	\section{Two Dimensions}
	Let 
	\begin{align}
		\vec{y} = A\vec{x} + \vec{n},
	\end{align}
	where 
	\begin{align}
		\vec{x} &\in \brak{\vec{s}_0,\vec{s}_1}, 
		\vec{s}_0 = \myvec{1 \\ 0},
		\vec{s}_1 = \myvec{0 \\ 1} \\
		\vec{n} &= \myvec{n_1 \\ n_2},
		n_1,n_2 \sim \gauss{0}{1}
	\end{align}
	
	\begin{enumerate}[label=\thesection.\arabic*,ref=\thesection.\theenumi]
	\item Plot 
	\begin{align}
		\vec{y}|\vec{s}_0 \text{ and } \vec{y}|\vec{s}_1
	\end{align}
	on the same graph using a scatter plot
	
	\solution Download the C source code to generate the data by executing the following commands
	\begin{lstlisting}
		wget https://github.com/Ankit-Saha-2003/AI1110/raw/main/Random-Numbers/codes/8.1.c
		wget https://github.com/Ankit-Saha-2003/AI1110/raw/main/Random-Numbers/codes/header.h
	\end{lstlisting}
	Compile and run the C program by executing the following
	\begin{lstlisting}
		cc -lm 8.1.c
		./a.out
	\end{lstlisting}
	
	Download the following Python code that plots Fig. \ref{fig-8.1} 
	\begin{lstlisting}
		wget https://github.com/Ankit-Saha-2003/AI1110/raw/main/Random-Numbers/codes/8.1.py
	\end{lstlisting}
	Run the code by executing
	\begin{lstlisting}
		python 8.1.py
	\end{lstlisting}
	
	\begin{figure}
		\centering
		\includegraphics[width=\columnwidth]{./figs/8.1.png}
		\caption{Plot of $\mbf{y}|\mbf{s}_0$ and $\mbf{y}|\mbf{s}_1$}
		\label{fig-8.1}
	\end{figure}
	
	\item For the above problem, find a decision rule for detecting the symbols $\mbf{s}_0 $ and $\mbf{s}_1$
	
	\solution Let
	\begin{align}
		\vec{y} = \myvec{y_1 \\ y_2}
	\end{align}
	
	Then, the line $y_2 - y_1 = 0$ seems to separate the two data sets. Thus,
	\begin{align}
		\hat{\vec{x}} = 
		\begin{cases}
			\vec{s}_0 & y_2 - y_1 < 0 \\
			\vec{s}_1 & y_2 - y_1 > 0
		\end{cases}
	\end{align}
	
	\item Plot 
	\begin{equation} 
		P_e = \pr{\hat{\vec{x}} = \vec{s}_1|\vec{x} = \vec{s}_0}
	\end{equation}
	with respect to the SNR from $0$ to $10$ $\mathrm{dB}$
	
	\solution Download the C source code to generate the data by executing the following commands
	\begin{lstlisting}
		wget https://github.com/Ankit-Saha-2003/AI1110/raw/main/Random-Numbers/codes/8.3.c
		wget https://github.com/Ankit-Saha-2003/AI1110/raw/main/Random-Numbers/codes/header.h
	\end{lstlisting}
	Compile and run the C program by executing the following
	\begin{lstlisting}
		cc -lm 8.3.c
		./a.out
	\end{lstlisting}
	
	Download the following Python code that plots Fig. \ref{fig-8.3} 
	\begin{lstlisting}
		wget https://github.com/Ankit-Saha-2003/AI1110/raw/main/Random-Numbers/codes/8.3.py
	\end{lstlisting}
	Run the code by executing
	\begin{lstlisting}
		python 8.3.py
	\end{lstlisting}
	
	\begin{figure}
		\centering
		\includegraphics[width=\columnwidth]{./figs/8.3.png}
		\caption{Plot of $P_e$}
		\label{fig-8.3}
	\end{figure}
	
	\end{enumerate}
\end{document}
