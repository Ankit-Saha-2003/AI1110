\documentclass[journal,12pt,twocolumn]{IEEEtran}

\usepackage{enumitem}
\usepackage{amsmath}
\usepackage{amssymb}
\usepackage{gensymb}
\usepackage{graphicx}
\usepackage{txfonts}         
\usepackage{listings}
\usepackage{lstautogobble}

\newcommand{\solution}{\noindent \textbf{Solution: }}
\providecommand{\pr}[1]{\ensuremath{\Pr\left(#1\right)}}
\providecommand{\brak}[1]{\ensuremath{\left(#1\right)}}
\providecommand{\cbrak}[1]{\ensuremath{\left\{#1\right\}}}
\providecommand{\mean}[1]{E\left[ #1 \right]}
\providecommand{\var}[1]{\mathrm{Var}\left[ #1 \right]}

\let\StandardTheFigure\thefigure
\numberwithin{equation}{section}
\renewcommand{\thefigure}{\theenumi}

\lstset {
	frame=single, 
	breaklines=true,
	columns=fullflexible,
	autogobble=true
}             
                               
\title{Random Numbers \\ \Large AI1110: Probability and Random Variables \\ \large Indian Institute of Technology Hyderabad}
\author{Ankit Saha \\ \normalsize AI21BTECH11004 \\ \vspace*{20pt} \normalsize  30 Jun 2022}


\begin{document}

	\maketitle
	
	\section{Uniform Random Numbers}
	Let $U$ be a uniform random variable between 0 and 1.
	\begin{enumerate}[label=\thesection.\arabic*,ref=\thesection.\theenumi]
	\item Generate $10^6$ samples of $U$ using a C program and save into a file called uni.dat

		\solution Download the C source code by executing the following command
		\begin{lstlisting}
			wget https://github.com/Ankit-Saha-2003/AI1110/raw/main/Random-Numbers/codes/1.1.c
		\end{lstlisting}
		Complie and run the C program by executing the following
		\begin{lstlisting}
			cc -lm 1.1.c
			./a.out
		\end{lstlisting}
	
	\item Load the uni.dat file into python and plot the empirical CDF of $U$ using the samples in uni.dat. The CDF is defined as
	\begin{align}
		F_{U}(x) = \pr{U \le x}
	\end{align}

	\solution  Download the following Python code that plots Fig. \ref{fig-1.2}
	\begin{lstlisting}
		wget https://github.com/Ankit-Saha-2003/AI1110/raw/main/Random-Numbers/codes/1.2.py
	\end{lstlisting}
	Run the code by executing
	\begin{lstlisting}
		python 1.2.py
	\end{lstlisting}
	\begin{figure}
		\centering
		\includegraphics[width=\columnwidth]{./figs/1.2.png}
		\caption{The CDF of $U$}
		\label{fig-1.2}
	\end{figure}
	
	\item Find a  theoretical expression for $F_{U}(x)$
	
	\solution The PDF of $U$ is given by
	\begin{align}
		P_{U}(x) = 
		\begin{cases}
			1 & x \in [0, 1] \\
			0 & \text{otherwise}
		\end{cases}
	\end{align}
	
	The CDF of $U$ is given by
	\begin{align}
		F_{U}(x) = \pr{U \le x} = \int_{-\infty}^x P_{U}(x) ~\mathrm{d}x
	\end{align}
	
	If $x<0$,
	\begin{align}
		\int_{-\infty}^x P_{U}(x) ~\mathrm{d}x = \int_{-\infty}^x 0 ~\mathrm{d}x = 0
	\end{align}
	
	If $c$,
	\begin{align}
		\int_{-\infty}^x P_{U}(x) ~\mathrm{d}x &= \int_{-\infty}^0 0 ~\mathrm{d}x + \int_0^x 1 ~\mathrm{d}x \\
		&= 0 + x \\
		&= x
	\end{align}
	
	If $x>1$,
	\begin{multline}
		\int_{-\infty}^x P_{U}(x) ~\mathrm{d}x \\= \int_{-\infty}^0 0 ~\mathrm{d}x + \int_0^1 1 ~\mathrm{d}x +  \int_1^x 0 ~\mathrm{d}x 
	\end{multline}
	\begin{align}
		\int_{-\infty}^x P_{U}(x) ~\mathrm{d}x &= 0 + 1 + 0 \\
		&= 1
	\end{align}
	
	Therefore, we obtain the CDF of $U$ as
	\begin{align}
		F_{U}(x) = 
		\begin{cases}
			0 & x < 0 \\
			x & 0 \le x \le 1 \\
			1 & x > 1
		\end{cases}
	\end{align}
	
	\item The mean of $U$ is defined as
	\begin{align}
		\mean{U} = \frac{1}{N}\sum_{i=1}^{N}U_i
	\end{align}
	and its variance as
	\begin{align}
		\var{U} = \mean{U- \mean{U}}^2 
	\end{align}
	Write a C program to  find the mean and variance of $U$
	
	\solution Download the C source code by executing the following command
	\begin{lstlisting}
		wget https://github.com/Ankit-Saha-2003/AI1110/raw/main/Random-Numbers/codes/1.4.c
	\end{lstlisting}
	Complie and run the C program by executing the following
	\begin{lstlisting}
		cc -lm 1.4.c
		./a.out
	\end{lstlisting}
	The output of the code is
	\begin{align}
		\mu_{\text{emp}} &= 0.500007 \\
		\mu_{\text{the}} &= 0.500000 \\
		\sigma_{\text{emp}}^2 &= 0.083301 \\
		\sigma_{\text{the}}^2 &= 0.083333
	\end{align}
	
	\item Verify your result theoretically given that
	\begin{align}
		\mean{U^k} = \int_{-\infty}^{\infty}x^k \mathrm{d}F_{U}(x)
	\end{align}
		
	\solution The mean of $U$ is given by
	\begin{align}
		\mean{U} = \int_{-\infty}^{\infty}x ~\mathrm{d}F_{U}(x) 
	\end{align}
	On differentiating the CDF of $U$, we get
	\begin{align}
		\mathrm{d}F_{U}(x) = 
		\begin{cases}
			0 & x < 0 \\
			\mathrm{d}x & 0 \le x \le 1 \\
			0 & x > 1
		\end{cases}
	\end{align}
	\begin{align}
		\therefore \mean{U} = \int_{0}^{1}x ~\mathrm{d}x = \frac12 = 0.5
	\end{align}
	
	Similarly,
	\begin{align}
		\therefore \mean{U^2} = \int_{0}^{1}x^2 ~\mathrm{d}x = \frac13
	\end{align}
	Now, the variance of $U$ is given by
	\begin{align}
		&\var{U} \\
		&= \mean{U- \mean{U}}^2 \\
		&= \mean{U^2 - 2U\mean{U} + (\mean{U})^2}
	\end{align}
	By linearity of expectation, we have
	\begin{align}
		&\mean{U^2} + \mean{-2U\mean{U}} + \mean{(\mean{U})^2} \\
		&= \mean{U^2} -2\mean{U}\mean{U} + (\mean{U})^2 \\
		&= \mean{U^2} - (\mean{U})^2 \\
		&= \frac13 - \brak{\frac12}^2 \\
		&= \frac{1}{12} \approx 0.083333
	\end{align}
	
	\end{enumerate}
	
\end{document}
