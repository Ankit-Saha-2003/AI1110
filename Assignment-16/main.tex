\documentclass{beamer}
\usetheme{CambridgeUS}

\setbeamertemplate{caption}[numbered]{}

\usepackage{amsmath}
\usepackage{amssymb}
\usepackage{gensymb}
\usepackage{graphicx}
\usepackage{txfonts}

\def\inputGnumericTable{}

\usepackage[latin1]{inputenc}                                 
\usepackage{color}                                            
\usepackage{array}                                            
\usepackage{longtable}                                        
\usepackage{calc}                                             
\usepackage{multirow}                                         
\usepackage{hhline}                                           
\usepackage{ifthen}
\usepackage{caption} 
\captionsetup[table]{skip=3pt}

\providecommand{\pr}[1]{\ensuremath{\Pr\left(#1\right)}}
\providecommand{\brak}[1]{\ensuremath{\left(#1\right)}}
\providecommand{\cbrak}[1]{\ensuremath{\left\{#1\right\}}}
\providecommand{\abs}[1]{\left\vert#1\right\vert}
\providecommand{\mean}[1]{E\left[ #1 \right]}

\newcommand{\myvec}[1]{\ensuremath{\begin{pmatrix}#1\end{pmatrix}}}
\newcommand{\mydet}[1]{\ensuremath{\begin{vmatrix}#1\end{vmatrix}}}

\let\vec\mathbf

\renewcommand{\thefigure}{\arabic{table}}
\renewcommand{\thetable}{\arabic{table}}                                     
                               
\title{AI1110 \\ Assignment 16}
\author{Ankit Saha \\ AI21BTECH11004}
\date{17 June 2022}


\begin{document}
	% The title page
	\begin{frame}
		\titlepage
	\end{frame}
	
	% The table of contents
	\begin{frame}{Outline}
    		\tableofcontents
	\end{frame}
	
	% The question
	\section{Question}
	\begin{frame}{Papoulis Exercise 11-7}
	Show that if $R_{xx}(\tau) = e^{-c\abs{\tau}}$, then the Karhunen-Loeve expansion of $X(t)$ in the interval $(-a,a)$ is the sum 
	\begin{align}
		\hat{X}(t) = \sum_{n=1}^{\infty} \brak{\beta_n b_n \cos \omega_n t + \beta_n' b_n' \sin \omega_n' t}
	\end{align}
	where
	\begin{align}
		&\tan a\omega_n = \frac{c}{\omega_n} && \cot a\omega_n' = -\frac{c}{\omega_n'} \\
		&\beta_n = (a + c\lambda_n)^{-\frac12} && \beta_n' = (a - c\lambda_n')^{-\frac12} \\
		&\mean{b_n^2} = \lambda_n = \frac{2c}{c^2 + \omega_n^2} && \mean{b_n'^2} = \lambda_n' = \frac{2c}{c^2 + \omega_n'^2}
	\end{align}
	\end{frame}
	
	% The proof	
	\section{Proof}
	\begin{frame}{Proof}
	The Karhunen-Loeve expansion of a process $X(t)$ is given by
	\begin{align}
		\hat{X}(t) = \sum_{n=1}^{\infty} c_n \varphi_n(t) \qquad 0 < t < T
	\end{align}
	where $\varphi_n(t)$ is a set of orthonormal functions in the interval $(0, T)$
	\begin{align}
		\int_0^T \varphi_n(t)\varphi_m(t)\mathrm{d}t = \delta[n-m]
	\end{align}
	and the coefficients $c_n$ are random variables given by
	\begin{align}
		c_n = \int_0^T X(t)\varphi_n(t)\mathrm{d}t
	\end{align}	 
	\end{frame}

	\begin{frame}
	Now, $R_{xx}(\tau) = e^{-c\abs{\tau}} \implies R_{xx}(t_1,t_2) = e^{-c\abs{t_1-t_2}}$ where $\tau = t_1-t_2$
	
	We now form the integral equation for a general $\varphi_n \triangleq \varphi$
	\begin{align}
		&\int_{-a}^a R_{xx}(t_1,t_2)\varphi(t_2)\mathrm{d}t_2 = \lambda\varphi(t_1) \\
		\implies &\int_{-a}^a e^{-c\abs{t_1-t_2}}\varphi(t_2)\mathrm{d}t_2 = \lambda\varphi(t_1) \\
		\implies &\int_{-a}^{t_1} e^{-c(t_1-t_2)}\varphi(t_2)\mathrm{d}t_2 + \int_{t_1}^{a} e^{c(t_1-t_2)}\varphi(t_2)\mathrm{d}t_2 = \lambda\varphi(t_1)
	\end{align}
	On differentiating with respect to $t_1$, we get
	\begin{align}
		\frac{\mathrm{d}}{\mathrm{d}t_1} \int_{-a}^{t_1} e^{-c(t_1-t_2)}\varphi(t_2)\mathrm{d}t_2 &= \frac{\mathrm{d}t_1}{\mathrm{d}t_1} e^{-c(t_1-t_1)}\varphi(t_1) + \int_{-a}^{t_1} \frac{\partial}{\partial t_1} \brak{e^{-c(t_1-t_2)}\varphi(t_2)}\mathrm{d}t_2 \\
		&= \varphi(t_1) -c\int_{-a}^{t_1} e^{-c(t_1-t_2)}\varphi(t_2)\mathrm{d}t_2
	\end{align}
	\end{frame}
	
	\begin{frame}
	Similarly, 
	\begin{align}
		\frac{\mathrm{d}}{\mathrm{d}t_1} \int_{t_1}^{a} e^{c(t_1-t_2)}\varphi(t_2)\mathrm{d}t_2 = -\varphi(t_1) + c\int_{t_1}^{a} e^{c(t_1-t_2)}\varphi(t_2)\mathrm{d}t_2
	\end{align}
	Thus,
	\begin{align}
		\lambda\varphi'(t_1) = -c\int_{-a}^{t_1} e^{-c(t_1-t_2)}\varphi(t_2)\mathrm{d}t_2 + c\int_{t_1}^{a} e^{c(t_1-t_2)}\varphi(t_2)\mathrm{d}t_2
	\end{align}
	\begin{multline}
		\implies \lambda\varphi''(t_1) = -c\brak{\varphi(t_1) -c\int_{-a}^{t_1} e^{-c(t_1-t_2)}\varphi(t_2)\mathrm{d}t_2} \\+ c\brak{-\varphi(t_1) + c\int_{t_1}^{a} e^{c(t_1-t_2)}\varphi(t_2)\mathrm{d}t_2}
	\end{multline}
	\begin{align}
		\implies \lambda\varphi''(t_1) = -2c\varphi(t_1) + c^2\brak{\int_{-a}^{t_1} e^{-c(t_1-t_2)}\varphi(t_2)\mathrm{d}t_2 + \int_{t_1}^{a} e^{c(t_1-t_2)}\varphi(t_2)\mathrm{d}t_2}
	\end{align}
	\end{frame}
	
	\begin{frame}
	Therefore, we have obtained the differential equation
	\begin{align}
		\lambda\varphi''(t_1) &= -2c\varphi(t_1) + c^2\lambda\varphi(t_1) \\
		\implies \varphi''(t_1) &= -\brak{\frac{2c}{\lambda} - c^2}\varphi(t_1) \\
		&= -\omega^2 \varphi(t_1)
	\end{align}
	Assuming $2c - \lambda c^2 \ge 0$, the solutions to this differential equation are $\beta \cos(\omega t_1 + \theta)$ (corresponding to $\lambda$) and $\beta' \cos(\omega' t_1 + \theta')$ (corresponding to $\lambda'$)
	for arbitrary constants $\beta$, $\theta$ and $\beta'$, $\theta'$
	\begin{align}
		\frac{2c}{\lambda} &- c^2 = \omega^2 \\
		\implies \lambda &= \frac{2c}{c^2+\omega^2} 
	\end{align}
	Similarly,
	\begin{align}
		\lambda' &= \frac{2c}{c^2+\omega'^2} 
	\end{align}
	\end{frame}

\end{document}